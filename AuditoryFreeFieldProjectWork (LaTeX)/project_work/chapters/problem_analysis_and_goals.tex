
% Problem Analysis and Goals

Bei akustischen Experimenten unter Freifeldbedingungen erfolgt die Präsentation der akustischen Signale in der Regel über Lautsprecher. Das Signal wird von der Versuchsperson entsprechend der Position des Lautsprechers im Raum wahrgenommen. Im virtuellen Setup werden die akustischen Signale von einer virtuellen Quelle erzeugt, deren Position im virtuellen Raum der des realen Lautsprechers entspricht. Die Wiedergabe erfolgt jedoch über einen Kopfhörer. Generell scheint ein Ton , der von einem Lautsprecher erzeugt wird, seinen Ursprung im Kopf zu haben. Nichts desto Trotz lässt sich durch Erzeugung eines virtuellen akustischen Raumes eine räumliche Lokalisation ermöglichen\footnote{siehe \cite{SLAS}}.
%Ziel ist die Simulation sämtlicher Eigenschaften der realen akustischen Umgebung mithilfe des
%virtuellen akustischen Raums.
