
% Discussion

Da das System nicht an Versuchspersonen getestet wurde lässt sich die Fähigkeit des Systems, die akustischen Eigenschaften eines Freifeldes zu simulieren, nur subjektiv bewerten.
Zwar ist die Leistung ausreichend um eine räumliche Lokalisation der Quellen im virtuellen Raum zu ermöglichen, jedoch ist keine Aussage über deren Präzision möglich.
Ein direkter Qualitätsvergleich zwischen VR-Setup und einem Freifeld lässt sich mit diesem Versuchsaufbau nicht durchführen und ist folglich nicht untersucht worden.\\
Ein entsprechender Vergleich wurde jedoch schon von Griffin D. Romigh, Douglas S. Brungart und Brian D. Simpson durchgeführt \footnote{siehe \cite{FFC}}. Hier wurde ein Versuchsaufbau verwendet der es ermöglicht die Versuchsperson unter beiden Bedingungen zu testen und somit einen objektiven Vergleich zu ziehen.\\
Während der Entwicklung wurde das Experiment zusätzlich mit unterschiedlichen Audio-Ausgabegeräten durchgeführt. Dies führte zu Verbesserungen in der räumlichen Wahrnehmung.
Die folgenden Varianten sind nach ihrer Qualität in steigender Reihenfolge angeordnet.
\begin{enumerate}
\item \textbf{HyperX Cloud II:} Kopfhörer, Over-Ear Headset, geschlossen
\item \textbf{Audio-Technica ATH-AD900X:}  High-Fidelity Kopfhörer, offen
\item \textbf{FX Audio DAC-X6:} DAC/AMP, in Kombination mit den oben genannten Kopfhörern
\end{enumerate}

Möglichkeit Nummer drei bietet zwar die beste Qualität aufgrund eines höheren Signal-Rausch-Abstandes ($>= 105 dB$) sowie einer besseren Kanalseparierung, bedarf jedoch zusätzlicher Adapter um das HTC-Vive mit einem DAC zu verbinden. Die beschränkte Länge von gängigen Adaptern(3.5mm Klinke auf USB) hat sich in Kombination mit der hier verwendeten Hardware als problematisch erwiesen. 