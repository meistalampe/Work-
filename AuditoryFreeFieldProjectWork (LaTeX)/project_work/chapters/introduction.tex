
% Introduction

Der Präzedenzeffekt beschreibt die räumliche Zuordnung einer Schallquelle anhand der ersten Wellenfront die das Ohr erreicht. Dabei werden alle Folgeereignisse, die in einem gewissen Zeitbereich nach der ersten eintreffen ebenfalls der selben Richtung zugeordnet. Der Effekt tritt dann auf wenn zwei gleiche Schallereignisse mit einer Verzögerungszeit von 2-30 ms am Ohr ankommen und ist zudem abhängig vom Pegel der Folgeereignisse.\footnote{siehe \cite{ency},S.140ff}\footnote{siehe \cite{handbuch},S.103ff}\\
Solche Folgeereignisse sind zum großen Teil Reflexionen, die von den Wänden des Raumes zurückgeworfen werden. Reflexionen, die das Ohr nach einer zu langen Verzögerungszeit erreichen werden als Echo oder separates Schallereignis wahrgenommen.\\
Aus diesem Grund werden akustische Messungen, die die Lokalisation von Schallquellen betreffen, nicht selten unter Freifeld-Bedingungen durchgeführt.
Um diese Bedingungen zu realisieren bedarf es einer umfangreichen Schallisolation des Messraums. Dieser Umstand verursacht zusätzliche Kosten und führt zu einer Beschränkung des Versuchsaufbaus.\\
Die virtuelle Realität bietet eine kostengünstige Möglichkeit verschiedenste Messbedingungen zu simulieren und damit wissenschaftliche Experimente durchzuführen.\\
Wie sich nun eine Freifeld-Umgebung zur Durchführung von akustischen Messungen in der virtuellen Realität realisieren lässt ist Thema dieser Projektarbeit. 


