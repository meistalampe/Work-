
% Zusammenfassung

Die virtuelle Realität ist ein High-End Benutzer-Computerinterface das hochauflösende Computer Grafik, moderne Tracking Technologien und Echtzeit Simulation nutzt umd eine virtuelle Welt zu erschaffen. Die Implementierung der virtuellen Realität in die Behandlung von Angststörungen, besonders im Bereich der Phobien, war ein zentrales Thema in der psyschologischen Forschung der letzten Jahrzehnte. Die Kombination aus herkömmlicher Konfrontationstherapie und virtueller Realität, ermöglicht die Durchführung einer Therapie innerhalb einer virtuellen Umgebung. Die so genannte Virtuelle Expositions Therapie bietet eine effiziente Alternative zur traditionellen in-vivo Therapie. Mit dem Ziel vorherige Therapiesysteme zu verbessern entwickelten wir ein in sich geschlossenes System zur Behandlung von Acrophobie mittels VR. Unser system zeichnet sich durch die Anpassungsfähigkeit der virtuellen Welt, die konstante Messung der psyschophysiologischen Angstreaktionen und deren Visualisierung in Echtzeit aus. In einer Experimentreihe haben wir 11 Versuchspersonen einem Höhenszenario innerhalb der virtuellen Umgebung ausgesetzt. Während der Exposition wurden EKG und Hautleitwert der Versuchspersonen gemessen und dem Nutzer angezeigt. Durch die Echtzeit-Präsentation ausgewählter Größen leiten wir wertvolle Informationen über den mentalen Zustand der Patienten an den Therapeuten weiter und unterstützen diesen bei der Entscheidungsfindung. Das Ziel dieser Arbeit war der Entwurf einer virtuellen Welt, die speziell auf die Anwendung zur Therapie von Höhenangst zugeschnitten ist und zu jeder Zeit and die Bedürfnisse des Patienten und des Therapeuten anpassen lässt. Darüber hinaus wurden, durch die Anwendung spezieller Signalverarbeitungsmethoden, die gemessenen Signale hinsichtlich ihres Informationsgehalts analysiert um der Entwicklung eines therapeutisches Werkzeugs zu assistieren. 