
% Introduction
In this chapter, we will give a introduction to the field of specific phobias and the associated therapy concept. The first part will contain fundamentals on phobias, exposure therapy and the effects of fear on the human body. Furthermore we will elaborate on biosignals containing information expressing these effects.\\
In the second part, we will 
 
\section{Motivation}
%- restrictions through phobias in daily life\\
%- subject of the study, reduction of acrophobia\\
%- the use a virtual environment can have in treating acrophobia\\
%- benefits of VRET over in vivo therapy (safety,one-fits all)\\

Suffering from a specific phobia such as acrophobia can be a huge interference with daily life. Those affected are often experiencing a slow progressing self-limitation fueled by their fear resulting in a declining quality of life. For people afflicted with acrophobia a therapy can therefor be life-changing.\\
Typically a therapy is designed to help patients face their fears and practice coping strategies resulting in reducing the fear altogether. A traditional exposure therapy can  be tainted with risks though. Major disadvantages being the amount of effort involved and dangers that come with giving therapy in extreme locations.\\
Virtual reality guided exposure therapy offers the possibility to treat patients in a controlled environment and eliminate the risk of injury. Furthermore virtual reality assisted treatment represents a safer and cost-efficient alternative that has great potential in improving phobia treatment based on its flexibility.\\
For example a in-vivo therapy consists of many different steps based on the initial extent of a patients fear and requires just as many individual stimulating situations. On the contrary one well designed virtual setup can easily adapt to all therapy stages and therefor be more convenient.\\



\section{Theoretical Background}

\subsection{Acrophobia} 
definition of fear and specific phobias,prevalence, evolutionary 	 purpose(fight or flight),connection fear to stress
  
\subsection{Stress}
definition of stress, ways of stress perception (eustress and distress)

\subsection{Galvanic Skin Response}
short explanation, influences(autonomic nervous system), role as method  to register physiological correlates of mental states like stress\\

- illustration of a typical gsr signal and explanation of its components (graph, peaks etc.) \\

- how and where is gsr usually measured? why there?\\

\subsection{Exposure Therapy}
what is exposure therapy?\\ 
when is it used? \\
how is it done?\\
what is needed for it to be successful? \\
how effective is it?\\

\section{General}
\subsection{State of the Art}
\subsection{Recent Advances in Research}
\section{Problem Analysis and Goals}
- analyse the problem with current models of exposure therapy
- show that my approach is different and how
- why my approach is better and makes sense
- goal is a safe and effective therapy option for acrophobia


