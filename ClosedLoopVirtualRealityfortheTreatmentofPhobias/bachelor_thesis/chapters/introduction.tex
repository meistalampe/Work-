
%% Introduction

%
%	Many suffer from phobias such as FoH ( statistics) 
%	life impairment
%	benefit of a therapy ( relate to r
%	obstacles in therapy in vivo
%	state that vr has been proven to be as effective as in vivo
%	and can be done on low budget
%	state advantages of vr over in vivo
%	vr has a by far lower refusal rate, more people can be treated

%Vertigo and nausea are only two of the symptoms many people are all too familiar with when confronted with height. Acrophobia or fear of heights is one of the most common anxiety disorders and is widespread in today's society.
%According to the work of Kapfhammer et al. (2014) roughly one third of the German population will at one point in their life be afflicted by visual height intolerance (vHI). Although  considering such high numbers it comes as a surprise that only about 11\% of susceptible individuals are willing to consult a doctor(Kapfhammer et al. 2014).
%Epidemiological research showing a peak in incidence around the second decade ( Huppert et al. 2012, Agras et al. 1969) and 




Virtual reality combines real-time computer graphics, body-tracking devices and high-resolution visual displays to create a computer-generated virtual environment. With their ability to immerse the user into a virtual mirror of the real world virtual environments are a powerful tool in clinical application, especially in the treatment of phobias (source). \\
Studies have shown anxiety disorders to be the most prevalent mental disorders (Kessler et al., 2005). Many consider exposure therapy the most effective form of treatment for specific phobias (DeRubeis and Crits-Cristoph,1998)( source ). However that may be, considering the nature of certain phobias such as fear of heights, exposure therapy involves a genuine risk of injury. Performing therapy in a virtual environment can therefore be a promising alternative to in vivo exposure.\\ 
The efficacy of virtual reality exposure therapy (VRET) has already been demonstrated in the past. A study conducted on acrophobia, comparing two groups of student subjects, showed that VRET is more effective than no treatment (Rothbaum et al.,1995). VRET was also found to be as effective as exposure in vivo in a more recent work by Emmelkamp et al. (2002). 
In addition to this using a virtual reality system can have a number of advantages. First and foremost being the ability to conduct therapy inside a controlled and secure environment like a therapist's office. This also implies therapy being less time consuming and provides financial benefits (Cavanagh and Shapiro, 2004). The possibility of having therapy in a more private scenario also could mean it becoming a more attractive choice for patients that are too anxious or fear public embarrassment. 


A recent study explored the acceptability of virtual reality exposure and in vivo exposure in subjects suffering from specific phobias supports this hypothesis. Seventy-six percent chose virtual reality over in vivo exposure. In addition to this the refusal rate of 3\% for virtual reality exposure was substantially lower than 27\% for in vivo exposure (Garcia-Palacios et al., 2007). 
These results suggest that virtual reality exposure could help increase the number of people who seek therapy for phobias. Although epidemiological studies show a lifetime prevalence of 28.5\% for vHI and 6.4\% for acrophobia alone, only 11\% of susceptible people consult a doctor (Huppert et al., 2013; Kapfhammer et al.,2015).\\
With so many people going untreated there is a distinct need to establish virtual reality in everyday clinical work.\\
In recent years there has been a lot of research on virtual reality treatment for different phobias trying just that.\\
-list studies for agora, aero phobia etc.\\

There also have been studies on ways to control the virtual reality. In a pilot study Levy et al. (2015) explored the possibility of a remote controlled virtual reality. After a trial session in a neutral virtual environment the patients received a total of six therapy sessions. The first three sessions were remote virtual reality exposure therapy (e-VRET) followed by three sessions in the presence of a therapist (p-VRET). The study showed that e-VRET not only is possible but produces results equal to p-VRET. This inevitably leads to the idea of an independent VRET. To our knowledge there has not yet been a form of VRET that does not depend on external control. A system that can adapt to the mental state of the patient throughout the entirety of therapy and therefore qualify for private use could help expanding the reach of exposure therapy even further.
Assessing the mental state of a patient is essential for the success of therapy. A task which usually falls to the hands of the therapist and in most cases relies on a verbal communication between both parties.
To ensure the quality of our therapy system we clearly have to provide some sort of substitute for this.\\
-stimulation triggers fear and this is shown through psychophysiological biosignals\\
-list studies that state gsr and hrv as valid indicators for emotions stress and fear\\
-hypothesis: independent system is possible if vr can trigger fear
-goal of this thesis is to design a virtual environment for this purpose and proving it is capable of 



we want to create a closed loop virtual reality, which is adaptive an can function without external control based on input from realtime measurement of psychophysiological data.

- establish that an adaptive VE would greatly benefit therapy
- past studies VR paradigms where none adaptive
- state goal as an automated 




Research Question
- is the designed VE able to elicit fear ( subjective)
- can the fear be related to changes in measured biosignals





%%%%%%%%%%%%%%%%%%%%%%%%%%%%%%%%%%%%%%%%%%%%%%%%%%%%%%%%%%%%%%%%%%%%%%%%%%%%%%%%%%%%%%%%
%In this chapter, we will give a introduction to the field of specific phobias and the associated therapy concept. The first part will contain fundamentals on phobias, exposure therapy and the effects of fear on the human body. Furthermore we will elaborate on biosignals containing information expressing these effects.\\
%In the second part, we will 
% 
%
%%- restrictions through phobias in daily life\\
%%- subject of the study, reduction of acrophobia\\
%%- the use a virtual environment can have in treating acrophobia\\
%%- benefits of VRET over in vivo therapy (safety,one-fits all)\\
%
%
%% write about epidemiology of phobias lifetime probability
%Suffering from a specific phobia such as acrophobia can be a huge interference with daily life. Those affected are often experiencing a slow progressing self-limitation fueled by their fear resulting in a declining quality of life. For people afflicted with acrophobia a therapy can therefor be life-changing.\\
%Typically a therapy is designed to help patients face their fears and practice coping strategies resulting in reducing the fear altogether. A traditional exposure therapy can  be tainted with risks though. Major disadvantages being the amount of effort involved and dangers that come with giving therapy in extreme locations.\\
%Virtual reality guided exposure therapy offers the possibility to treat patients in a controlled environment and eliminate the risk of injury. Furthermore virtual reality assisted treatment represents a safer and cost-efficient alternative that has great potential in improving phobia treatment based on its flexibility.\\
%For example a in-vivo therapy consists of many different steps based on the initial extent of a patients fear and requires just as many individual stimulating situations. On the contrary one well designed virtual setup can easily adapt to all therapy stages and therefor be more convenient.\\
%
%
%
%\section{Theoretical Background}
%
%\subsection{Acrophobia} 
%definition of fear and specific phobias,prevalence, evolutionary 	 purpose(fight or flight),connection fear to stress
%  
%\subsection{Stress}
%definition of stress, ways of stress perception (eustress and distress)
%
%\subsection{Galvanic Skin Response}
%%short explanation, influences(autonomic nervous system), role as method  to register physiological correlates of mental states like stress\\
%%
%%- illustration of a typical gsr signal and explanation of its components (graph, peaks etc.) \\
%%
%%- how and where is gsr usually measured? why there?\\
%
%%- EDA is sensitive to psychological states and processes, easy to measure
%%- exosomatic method of recording skin conductance level and skin conductance response as method of choice ( Fowles et al. 1981)
%%- secretory theory, by tarchanoff relates EDA to sweat gland activity, supported through Darrow 1927 who showed a close relation of the sweat secretion and EDA
%%-scr begins about 1 s earlier than moisture would appear on the skin surface -> gland activity not sweat on the skin itself is critical for EDA
%%- palmar sweat glands are innervated by the sympathetic chain of the autonomic nervous system -> EDA reflects sympathetic activation
%%- In fact, we would date the beginning of
%%the modern era of EDA research to the early 1970s when
%%Lykken and Venables proposed standardized techniques
%%of recording skin conductance and standardized units of
%%measurement
%%- Another issue of central importance concerns the psychological
%%significance of EDA. From the beginning, this
%%response system has been closely linked with the psychological
%%concepts of emotion, arousal, and attention.
%%-“Every stimulus accompanied by an emotion produced
%%a deviation of the galvanometer to a degree in direct
%%proportion to the liveliness and actuality of the emotion
%%aroused” (Peterson, 1907, cited by Neumann & Blanton,
%%1970, p. 470)
%%- Woodworth and Schlosberg ,1954: They supported this indexing relationship by noting that tonic SCL is generally low during sleep and
%%high in activated states such as rage or mental work. The
%%authors also related phasic SCRs to attention, noting that
%%such responses are sensitive to stimulus novelty, intensity,
%%and significance.
%%- Anatomical and physiological basis:
%%- There are two forms of sweat glands in the human
%%body: the apocrine, which have been less studied, and
%%the eccrine, which have been of primary interest to psychophysiologists.
%%The primary function of most eccrine
%%sweat glands is thermoregulation. However, those located
%%on the palmar and plantar surfaces are thought to be more
%%related to grasping behavior than to evaporative cooling
%%(Edelberg, 1972a) and they have been suggested to be more
%%responsive to psychologically significant stimuli than to
%%thermal stimuli.\\
%
%
%\subsection{Exposure Therapy}
%what is exposure therapy?\\ 
%when is it used? \\
%how is it done?\\
%what is needed for it to be successful? \\
%how effective is it?\\
%
%\section{General}
%\subsection{State of the Art}
%\subsection{Recent Advances in Research}
%\section{Problem Analysis and Goals}
%- analyze the problem with current models of exposure therapy
%- show that my approach is different and how
%- why my approach is better and makes sense
%- goal is a safe and effective therapy option for acrophobia
%
%
