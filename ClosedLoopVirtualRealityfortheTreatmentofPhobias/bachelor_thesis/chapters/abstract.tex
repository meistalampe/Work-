
% Abstract
Virtual reality is a high-end user-computer interface that combines real-time computer graphics, body-tracking devices and high-resolution visual displays to create a computer-generated virtual environment. Its implementation into the treatment of anxiety disorders, such as phobias, has been studied extensively over the past decades. \\
The combination of exposure therapy and virtual reality environments, called virtual reality exposure therapy, provides for an efficient alternative to traditional in-vivo exposure. In an attempt to improve upon virtual reality systems that have been used in the past, we created a closed loop virtual reality system that features an adaptable virtual reality environment and real-time information visualization of sympathetic fear responses.\\
We have designed our virtual reality environment specifically for the treatment of acrophobia for its intended use in a psychological study on the matter.\\
We conducted an experiment in which we exposed 11 subjects exposed to a visual height challenge inside the virtual reality environment. During the exposure, the skin response level as well as heart-rate variation of the subjects were measured and displayed to the user, controlling the virtual reality. By presenting the extracted features, we relay crucial information about the patient's current mental state and provide a support tool for the therapist's decision making and therefore a possible improvement to the quality of virtual exposure therapy. \\
The objective of the present thesis, other than designing the virtual reality environment, was to provide for signal processing and feature extraction to assist in the creation of a classification tool.   


%%%%%%%%%%%%%%%%%%%%%%%%%%%%%%%%%%%%%%%%%%%%%%%%%%%%%%%%%%%%%%%%%%%%%%%%%%%%%%%%%%%%%%%%%%%%%%%%%%%%%%%%
%- a brief mentioning of the study and its attempt to treat acrophobia with a %virtual environment\\
%- theme of this thesis: the design of a VR fit to treat patients\\


%start
%Virtual reality is a high-end user-computer interface that involves real-time simulation and interactions through multiple sensorial channels\footnote{\citet*{burdea2003virtual}}. Virtual reality devices have been used for years in industry and healthcare, where researchers have been trying to develop and implement virtual reality in ways that could assist approaching and training various situations.\\
%With the advent of commercial virtual reality technology Head-Mounted-Displays (HMD) provide easier access to a vast field of possible applications. Including medical research using virtual realty to diagnose and treat diseases, allowing not only professionals but rather the patients to interact with a virtual environment.\\
%A striking example can be found in the field of psychotherapy.
%The treatment of psychological conditions such as acrophobia can be a challenging task considering traditional therapy involves exposing the patient to a hostile situation.\\
%As part of a study, which is set to evaluate the effectiveness of virtual reality guided exposure therapy in comparison to the established in-vivo therapy, the goal of this thesis is to provide a closed loop virtual reality fit to treat acrophobia.\\
%On that account we created a virtual environment which is appropriate for treating different degrees of acrophobia and designed a setup capable of adapting to the users needs. We demonstrate that a well tailored virtual reality can provide a safe and controlled environment without compromising on the quality of the therapy.\\


%Exposure therapy

%One treatment for patients with phobias is exposure therapy. In one instance, psychiatrists at the University of Louisville are using VR to help patients deal with fears of things like flying and claustrophobia.
%
%The VR experiences provide for a controlled environment in which patients can face their fears and even practice coping strategies, as well as breaking patterns of avoidance — all while in a setting that's private, safe, and easily stopped or repeated, depending on the circumstances. 


%furthermore our virtual can be controlled by a therapist 
%the therapist will be able to exercise control through a matlab program 
%provided with real time visual data the therapist will be a substitute for the AI which will later be controlling 
%the vr respectively to the measured data
%
%the vr and the related pc will feed visual input to the subject 
%this input is processed by the subject and he gives output information in the form of gsr and heart rate
%serving as input for our third system, the therapist (visual presentation of processed data)
%therapist can control vr ---loop closed