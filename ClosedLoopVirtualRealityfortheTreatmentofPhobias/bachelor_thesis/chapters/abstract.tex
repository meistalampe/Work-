
% Abstract
Virtual reality is a high-end user-computer interface that involves real-time simulation and interactions through multiple sensorial channels. Its implementation into the treatment of anxiety disorders, such as phobias, has been studied extensively over the past decades. The combination of efficacious in-vivo exposure therapy and virtual reality environments provides for an efficient alternative to traditional approaches. In an attempt to improve upon virtual reality systems that have been used in the past, we created a closed loop virtual reality system that features an adaptable virtual reality environment and real-time measures of sympathetic fear responses. We have designed our virtual reality environment specifically for the treatment of acrophobia for it will be used in a psychological study on the matter of virtual reality exposure therapy. In an preceding experiment, X subjects were exposed to the virtual environment

, to prove the functionality of our system and . The main objective of the present thesis was

 However, there is still room for improvements, as experiments, conducted to this day, lack the ability of adapting on the fly. Therefore, we have designed a closed loop virtual reality system that is able to adapt to the immediate needs of both the therapist and the patient. Preceding its deployment in a 


%Suffering from a specific phobia such as acrophobia can be a huge interference with daily life. Those affected are often experiencing a slow progressing self-limitation fueled by their fear resulting in a declining quality of life. For people afflicted with acrophobia a therapy can therefor be life-changing.\\
%Typically a therapy is designed to help patients face their fears and practice coping strategies resulting in reducing the fear altogether. However, conducting a traditional exposure therapy can be particularly risky for the patient.
%Virtual reality guided exposure therapy offers the possibility to treat patients in a controlled environment and eliminate the risk of injury. Furthermore virtual reality assisted treatment represents a safer and cost-efficient alternative that has great potential in improving phobia treatment based on its flexibility.\\
%For example a in-vivo therapy consists of many different steps based on the initial extent of a patients fear and requires just as many individual stimulating situations. On the contrary one well designed virtual setup can easily adapt to all therapy stages and therefor be more convenient.\\





%%%%%%%%%%%%%%%%%%%%%%%%%%%%%%%%%%%%%%%%%%%%%%%%%%%%%%%%%%%%%%%%%%%%%%%%%%%%%%%%%%%%%%%%%%%%%%%%%%%%%%%%
%- a brief mentioning of the study and its attempt to treat acrophobia with a %virtual environment\\
%- theme of this thesis: the design of a VR fit to treat patients\\


%start
%Virtual reality is a high-end user-computer interface that involves real-time simulation and interactions through multiple sensorial channels\footnote{\citet*{burdea2003virtual}}. Virtual reality devices have been used for years in industry and healthcare, where researchers have been trying to develop and implement virtual reality in ways that could assist approaching and training various situations.\\
%With the advent of commercial virtual reality technology Head-Mounted-Displays (HMD) provide easier access to a vast field of possible applications. Including medical research using virtual realty to diagnose and treat diseases, allowing not only professionals but rather the patients to interact with a virtual environment.\\
%A striking example can be found in the field of psychotherapy.
%The treatment of psychological conditions such as acrophobia can be a challenging task considering traditional therapy involves exposing the patient to a hostile situation.\\
%As part of a study, which is set to evaluate the effectiveness of virtual reality guided exposure therapy in comparison to the established in-vivo therapy, the goal of this thesis is to provide a closed loop virtual reality fit to treat acrophobia.\\
%On that account we created a virtual environment which is appropriate for treating different degrees of acrophobia and designed a setup capable of adapting to the users needs. We demonstrate that a well tailored virtual reality can provide a safe and controlled environment without compromising on the quality of the therapy.\\


%Exposure therapy

%One treatment for patients with phobias is exposure therapy. In one instance, psychiatrists at the University of Louisville are using VR to help patients deal with fears of things like flying and claustrophobia.
%
%The VR experiences provide for a controlled environment in which patients can face their fears and even practice coping strategies, as well as breaking patterns of avoidance — all while in a setting that's private, safe, and easily stopped or repeated, depending on the circumstances. 


%we want to find out if it is possible to design a fully automatic therapy system using vr and psycho physiological measurement.
%therefor we will test our virtual environment with a random group of subjects
%and measure ecg and gsr
%the goal of the conducted experiment is to show that our virtual reality is capable ob causing fear
%this will be done by evaluating the measured bio data 
%
%furthermore our virtual can be controlled by a therapist 
%the therapist will be able to exercise control through a matlab program 
%provided with real time visual data the therapist will be a substitute for the AI which will later be controlling 
%the vr respectively to the measured data
%
%the vr and the related pc will feed visual input to the subject 
%this input is processed by the subject and he gives output information in the form of gsr and heart rate
%serving as input for our third system, the therapist (visual presentation of processed data)
%therapist can control vr ---loop closed