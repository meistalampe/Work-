
% Materials and Methods
%6. Summary of Publication Recommendations
%Published reports must contain sufficient detail to allow other
%experimenters to try to replicate the reported results and to help
%explain failures of replication. Given the limited space that journals
%provide, recommendations will be summarized below for both
%necessary and optional recording and environmental procedures to
%be reported in publications of studies using EDA.
%First and above all, the method of measurement has to be specified:
%endosomatic or exosomatic, direct or alternating current (if
%any) applied to the skin, and constant voltage or constant current.
%The applied voltage (or current) must be noted. If commercially
%available instrumentation has been used, the manufacturer and
%instrument type should be mentioned. Furthermore, calibration
%procedures should be specified.
%Second, methods of signal conditioning and storage need to be
%specified, including procedures for separating EDL from EDRs,
%if applied, time constants of amplifiers, separate grounding procedures,
%if used, A/D conversion rate, and sampling frequency
%(for EDL and EDRs if stored separately).
%Third, recording sites should be specified for active and inactive
%electrodes (if applicable). If the sites were pretreated, the procedure
%should be reported in detail. Also essential is providing details for
%electrodes and electrolytes that were used, such as electrode metal
%(e.g., sintered Ag/AgCl), area of contact (either in square centimeters
%or diameter), method of fixation (e.g., double-sided adhesive
%tape), details of the used electrolyte, such as type of gel or base,
%ionic type and concentration (e.g., 0.08 M or 0.5% NaCl), or, in the
%case of disposable electrodes, brand and type plus as much of the
%above mentioned information as is available from the manufacturer. It is important to know how long electrodes were attached
%before the recording started and how long they stayed in place.
%Details of how polarization was controlled and how electrodes
%were stored should be given if available. In the case of DC recording,
%we recommend using a polarity reversal switch between segments
%within a session.
%Fourth, signal evaluation needs to be reported in detail, whereas
%the sampling rate (for tonic and phasic measures separately, if
%different) and specification of time windows for tonic and phasic
%measures (e.g., latency windows for EDR onset being 1–4 s after
%stimulus onset) are mandatory. For EDRs, a minimum amplitude
%criterion must be specified and reported (e.g., 0.01 mS for SCRs to
%be scored). The standard terminology mentioned in Section 3
%should be adhered to. The term EDR magnitude should be reserved
%for the average amplitude calculated from a series of responses that
%include zero amplitudes. Any treatment of superimposed EDRs
%should be specified. Methods of detection and elimination of
%recording artifacts should be described if applicable.
%Besides the usual details to be reported about procedures for
%laboratory and field settings, it is important for EDA measurement
%to specify baseline conditions in detail, including length and statistical
%treatment during EDAdata evaluation. The gender, age, and
%ethnicity of the participants (e.g., number, range or mean, and
%standard deviation) are essential for comparison with EDA results
%from other studies. Medication or drug use (including caffeine
%intake before participation in the study) need to be reported as well.
%Clothing as well as inside and outside temperatures and their possible
%changes during the recording periods should be reported in as
%much detail as possible. If available (e.g., in case of room airconditioning),
%relative humidity should be reported as well.

\section{Materials}
mention the SNNU and the lab where the study takes place


\subsection{Setup}
- description of the therapy setup\\
- graphic 1, shows a patient inside the defined treatment area, wearing VR-Headset, the lighthouse system, eeg and gsr sensors, connection to the pc controlled by the physician


\subsection{Procedure(Paradigm)}
- how many subjects did participate?\\
- which tasks did the patients fullfill? (cross the bridge etc.)\\
- duration of the experiment\\

- description of the virtual environment, the procedure (baseline measurement,VRET in detail)\\ 
- pictures that show the VE in it's starting state as well as it's therapy state (descended floor)\\
- description of how the VR is controlled by the user(which parameters can be influenced)


\section{Methods}
- main objective is the measurement of gsr during the therapy and the evaluation of the gsr data concerning the stress of the patient during the therapy\\
- how is the gsr information processed and evaluated?\\
how is it presented to the user?\\
- description of how the VR is controlled by the user(which parameters can be influenced)
- graphic of control chain
