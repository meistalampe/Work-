
% Discussion
\section{Data Acquisition Quality}
We have obtained 22 sets of data, 11 for each EDA and ECG, of the 11 subjects that were measured in our experiment. Three EDA measures were rejected due to exceeding the imaging range of the BITalino device of 0-25 $\mu S$. The 3 data sets correlate with the subjects that stated increased palmar sweating, in the beginning of the experiment. We also had to eliminate 1 ECG measure due to immense motion artifacts, which although expected to some degree were to frequent and likely caused by contact of the electrode cables to the subject's legs. To ensure higher efficiency in future measures involving the BITalino (r)evolution it is recommended to increase attention on cable management and reconsider palmar electrode placement for ECG measurements, which are particularly prone to motion artifacts. 

\section{Sympathetic Activation}
Earlier, we have highlighted the connection between sympathetic activation and psychophysiological measures, particularly heart rate variation and tonic components of the EDA. It is suggested that an activation of the sympathetic nervous system is caused by the presence of a stimulus and reflected by an increase of these measures. We have chosen to indicate the heart rate in the form of RR intervals. Thus, an increase in heart rate, caused by a stimulus, is equal to a decrease in RR distance. As shown in figure \ref{ECGbpImg} the exposure has caused an overall decrease in RR distance in all subjects. This effect can also be found in figure \ref{ECGhistoImg}, where the RR distribution of, both, the baseline and the exposure measurement of a single subject (see \ref{ECGbpImg}, subject number 6) was mapped in the form of a histogram, to illustrate the drift towards lower RR distances during the exposure. In addition the average heart rate, which is shown on top of each histogram (see \ref{ECGhistoImg}) increase significantly.
The same effect could be observed in EDA measures (see \ref{EDAbpImg}, \ref{EDAhistoImg}), as the average peak interval time decrease and the peak rate of NS-SCRs increased. When comparing the average differences in SCL between the two phases, we found an observable increase during the exposure (see \ref{SCLbpImg}). Therefore, we are able to conclude our virtual environment to be sufficiently stimulating and the offered stimulus are considered effective. 

\section{Statistic Evaluation and Feature Quality}

To evaluate the possible use of the extracted features we have conducted a two-sample t-test for paired samples. As mentioned earlier, this test was designed around the null hypothesis $H0: \mu_{X}-\mu_{Y} = \omega_{0}$, where we assume $\omega_{0}=0$ and therefore impute that there will be no difference between two measures of the same subject. This hypothesis has been disproved for SCL and RR interval measurements. Therefore we have shown that there is a significant increase in these features during the exposure. In addition, we gathered information on effect size of the t-test, using the Cohen's d, and therefore the practical significance of a significant difference in the mean values. With d values of 1.4556 for SCL and 0.9010 for RR distance we distinctly exceed the criteria of a strong effect (d > 0.8). This means that an increase in both signal features is likely to be observed. Thus, we acknowledge the quality of the average change in SCL and the average RR interval for the real-time assessment of an individual's mental state, particularly the sympathetic fear reaction.\\
There are however some conditions to the application of the two-sample t-test. The most important one being that the samples are derived from populations with normal distribution. Thus, we performed a test on normality with the null hypothesis H0, which states that population of the tested sample possesses a normal distribution. We have confirmed H0 for all both ECG and SCL and therefore assume the results of the t-test to be justified. On the other hand the t-test for EDA peak intervals yielded positive results (this means H0 was confirmed). Therefore no significant difference between baseline and exposure was found. How strongly this is effected by the associated baseline sample failing the Shapiro-Wilk test and the time difference of the two measurements could not be determined. However, the test results suggest the elimination of average peak interval as a quality feature in the detection of sympathetic activation during the exposure. 
The evaluation of the correlationcoefficient r yielded only neglectable effect sizes for all three data pairs.
 